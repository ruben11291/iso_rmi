\documentclass[a4paper,11pt,oneside]{article}
\usepackage[pdftex]{graphicx}
\usepackage[utf8]{inputenc}
\usepackage[spanish]{babel}
\usepackage{fancyhdr}
\usepackage[usenames,dvipsnames]{color}
\usepackage{colortbl}
\usepackage{tabularx}
\usepackage{hyperref}
\usepackage{pdfpages}
\usepackage{lscape}


\setlength{\headheight}{15.2pt}
\pagestyle{fancy}

\lhead{\nouppercase{\leftmark}}
\chead{}
\rhead{}

\hypersetup{
colorlinks,
citecolor=black,
filecolor=black,
linkcolor=black,
urlcolor=black
}

\setlength{\parskip}{6pt}

\begin{document}

%%%% Title Page %%%%%

%\pagenumbering{alph}

\begin{titlepage}
\begin{center}

% Upper part of the page
\includegraphics[width=0.2\textwidth]{img/logo-uclm.png}\\[1cm]
\textsc{\LARGE Escuela Superior de Informática}\\[0.5cm]
\textsc{\Large Universidad de Castilla -- La Mancha}\\[2.5cm]

% Title
{\LARGE Práctica 2}\\[0.5cm]
\rule{\linewidth}{0.5mm}\\[0.4cm]
{\huge \textbf{Tres en Raya Definitivo Web}}\\
\rule{\linewidth}{0.5mm}\\[0.4cm]

% Authors
\begin{minipage}{0.5\textwidth}
\begin{flushleft}
\large
\hspace{1cm}\textbf{\emph{Autores}}\\
Marchán Loro, Javier\\
Peralta López, Ángel\\
Pérez Pascual, Rubén\\
Ruedas García, Antonio\\
\end{flushleft}
\end{minipage}
\vfill

% Bottom of the page
\begin{minipage}{\textwidth}
\large
\begin{tabular}{rl}
\textbf{Asignatura}: & Ingeniería del Software II\\
\textbf{Titulación}: & Ingeniería Informática\\
\textbf{Fecha}: & \today
\end{tabular}
\end{minipage}

\end{center}
\end{titlepage}

%%%% end Title Page %%%%%

\pagenumbering{roman}
\setcounter{page}{1}

\tableofcontents
\clearpage

\pagenumbering{arabic}
\setcounter{page}{1}

\clearpage
\section{Introducción}

Este documento describe el análisis, desarrollo e implementación llevados a cabo para obtener
el juego \textbf{Tres en Raya Definitivo Web} \footnote{Este documento no explica las reglas del juego}.

Se trata de una aplicación cliente-servidor web. Es requisito indispensable la existencia y ejecución del servidor de escritorio RMI ya que el servidor web depende de este.
El lenguaje utilizado para la implementación ha sido Java y el framework \emph{Google Web Toolkit} (GWT).

Este documento pretende que el lector obtenga una idea clara de la composición del sistema, para ello se muestra la arquitectura del sistema que ayudarán al lector a hacerse una imagen mental, a alto nivel, del sistema desarrollado.
Tras esto, se explicarán las pruebas realizadas.

\clearpage


\section{Arquitectura del sistema}

En la Figura \ref{arq_sistema} se muestra una visión de la arquitectura del sistema. Se
trata de una arquitectura cliente-servidor a través de servicios web. Su implementación se ha llevado a cabo usando el framework Google Web Toolkit(GWT).
El servidor proporciona e implementa una serie de operaciones a través de un interfaz que extiende de ``Remote Service''. Este servidor web es un mero proxy para los clientes web, ya que no contiene la funcionalidad del servidor como tal, que la sigue mantiendo el servidor de escritorio.

 \begin{figure}[h]
 \centering
 \includegraphics[scale=0.65]{img/arq.png}
 \caption{Interacción entre cliente web, servidor web y servidor de escritorio}
 \label{arq}
 \end{figure}

Los clientes solicitan los servicios que el servidor web ofrece usando la interfaz proporcionada y este, se comunica a través de RMIcon el servidor de escritorio. En esta comuncación el servidor web actúa en rol de cliente del servidor de escritorio, que es el componente que contiene toda la lógica propia del servidor como tal. Esta comunicación se realiza mediante RMI.
Por lo tanto los flujos de información que se producen, se resumen en:
\begin{itemize}
 \item Clientes web a servidor web.
\item Servidor web con rol de cliente de escritorio a servidor de escritorio.
\end{itemize}
En la figura \ref{arq_sistema} se puede observar la arquitectura descrita.
Al igual que en la anterior práctica, la lógica del juego reside en la parte del cliente web.

En la siguiente figura, se puede apreciar la arquitectura de la aplicación web, conformada tanto por el servidor web y el cliente web.

 \begin{figure}[h]
 \centering
 \includegraphics[scale=0.45]{img/diagrama.png}
 \caption{Arquitectura del sistema}
 \label{arq_sistema}
 \end{figure}

\subsection{Arquitectura del Servidor web}

Como se puede apreciar en la Figura \ref{arq_sistema}, el servidor web está basado únicamente en una capa de comunicación ya que no posee dominio o persistencia alguna.

\begin{itemize}
  \item \emph{Comunicación:} La clase encargada de esta función es \emph{ServerImpl}. Implementa la interfaz \emph{Servidor} e \emph{ICliente}. Al extender de ``Remote Service'' proporciona servicios a los clientes web que los soliciten. Cuándo estos son solicitados, se envían al servidor de escritorio para que realmente sirva sus peticiones. Este envío es realizado mediante RMI. Para recibir la respuesta necesita implementar la interfaz \emph{ICliente} ya que el servidor de escritorio desconoce que es web, simplemente se trata como otro cliente cualquiera.
 del paquete de dominio.
\end{itemize}

\subsection{Arquitectura del Cliente Web}

En la Figura \ref{arq_sistema} se muestra la arquitectura multicapa del servidor:

\begin{itemize}
 \item \emph{Comunicación:} La clase encargada de estas tareas es \emph{UltimateTicTacToeWeb}. Simplemente realiza llamadas asíncronas al servidor web y obtiene su respuesta.
  \item \emph{Interfaz:} Contiene las funciones que representan y modifican la GUI del cliente en el navegador. Se encarga la clase \emph{UltimateTicTacToeWeb} de esta función.
 \item \emph{Dominio:} Contiene las clases de dominio como las encargadas de representar a cualquier juego. Estas clases son Tablero3x3, Tablero9x9 y Jugador.
\end{itemize}

\clearpage

\section{Pruebas de los casos de uso}

\subsection{Pruebas para el servidor}

Las pruebas para el sistema del servidor se han hecho de tal forma que puedan ser ejecutadas de forma independiente. No necesita ningún cliente. El paquete test.communications contiene un cliente construido específicamente para las pruebas y consiste en un modelo simplificado de la clase Cliente del sistema cliente.

\begin{itemize}
\item Paquete \emph{src.server.tests}

\begin{center}
{\footnotesize
\begin{tabularx}{0.95\textwidth}{p{0.2\textwidth}|X}

\textbf{Caso de Uso} & \textbf{Test} \\
& \\
Registrar Usuario & TestDAOJugador\\
Autenticar & TestDAOJugador\\
Cerrar Sesión & TestCerrarSesion\\
Registrar moviminento & TestDAOMovimiento\\
Enviar movimiento a jugador & TestReto\\
Enviar respuesta a reto & TestReto\\
\end{tabularx}
}
\end{center}

Los test no prueban la comunicación entre clientes, solo la funcionalidad del servidor a partir de la fachada del servidor (clase FTERD, ver Diagrama de clases del servidor).

\item Paquete \emph{src.server.tests.communications}.
\begin{center}
{\footnotesize
\begin{tabularx}{0.95\textwidth}{p{0.2\textwidth}|X}
\textbf{Caso de Uso} & \textbf{Test} \\
& \\
Cerrar sesión & TestCerrarSesion\\
Registrar usuario & TestRegistrarYAnyadirJugador\\
Autenticar & TestRegistrarYAnyadirJugador\\
Enviar respuesta a reto & TesteRetoRechazar\\
Enviar respuesta a reto & TestRetoAceptar\\
\end{tabularx}
}
\end{center}
\end{itemize}

\clearpage

\subsection{Pruebas para el cliente}

Las pruebas del cliente son dependientes del servidor. Es decir, el test FTERDTest implica que se esté ejecutando el servidor para la posible evaluación del ``test-case''. Los demás ficheros de test no dependen de la ejecución del servidor, son independientes.


\begin{center}
{\footnotesize
\begin{tabularx}{0.95\textwidth}{p{0.2\textwidth}|X}

\textbf{Caso de Uso} & \textbf{Test} \\
& \\
Registrar & FTERDTest\\
Iniciar sesión & FTERDTest\\
Cerrar sesión &FTERDTest \\
Salir de la partida & FTERDTest\\
Retar jugador &FTERDTest \\
Aceptar reto&FTERDTest\\
Realizar movimiento& Tablero9x9Test\\
\end{tabularx}
}
\end{center}

Existen también los test Tablero3x3Test, JugadorTest y Tablero9x9Test que evalúan los posibles escenarios del juego. Esto es, se evalúan las posibles situaciones de movimientos sobre el tablero y se comprueban los resultados de la ejecución.
\begin{center}
{\footnotesize
\begin{tabularx}{0.95\textwidth}{p{0.2\textwidth}|X}

\textbf{Test} & \textbf{Función} \\
& \\
Tablero3x3Test & Evalúa que se coloquen las fichas en el lugar correcto y el tablero empatado y ganado.\\
Tablero9x9Test & Evalúa la correcta colocación de las fichas según las reglas del juego y que la victoria de los tableros, el empate total y la victoria del juego se realizen correctamente. \\
JugadorTest &Evalúa que el jugador que pone una ficha, realmente ejecuta todo el procedimiento y coloca la ficha, si las reglas lo permiten. \\
\end{tabularx}
}
\end{center}
\clearpage

\end{document}
