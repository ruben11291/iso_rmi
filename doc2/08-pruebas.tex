\section{Pruebas de los casos de uso}

\subsection{Pruebas para la aplicación web}

Los test de la aplicación web han sido llevado a cabo mediante el testbed \emph{Selenium}. Las pruebas que han podido ser cubiertas y reproducibles han sido:

\begin{itemize}
\item Pruebas relacionadas con el login, posibles causas de error y de éxito.
\item Pruebas relacionadas con el registro, posibles causas de error y de éxito.
\item Pruebas relacionadas con el cierre de sesión, posibles causas de error y de éxito.
\end{itemize}

Para poder realizar estas pruebas ha sido necesario introducir una clase \emph{Broker} para la comunicación con la base de datos y una clase auxiliar que contiene las operaciones de borrado de esta base de datos.

Las pruebas relacionadas con la lógica de dominio por parte del cliente quedan cubiertas con las pruebas que se realizaron para el cliente web, ya que el código se ha cogido tal cual de esa implementación.

Se ha intentado realizar una prueba respecto de la jugabilidad, pero al no ser reproducible, se ha descartado. Estas pruebas han sido llevadas a cabo de forma manual con mucha dedicación y esfuerzo, probando las múltiples posibilidades para poder llevar la cobertura al máximo posible.

Las pruebas se encuentran en el paquete \emph{test.terd.web}.
