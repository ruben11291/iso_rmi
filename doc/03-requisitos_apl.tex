% \section{Requisitos de interfaces externas}

% \section{Requisitos funcionales}

\subsection{Registrar Usuario}

{\footnotesize
\begin{tabularx}{0.95\textwidth}{p{0.2\textwidth}|X}

\textbf{Caso de uso} & Registrar Usuario \\

\textbf{Precondición} & Ninguna \\

\textbf{Escenario general} & \begin{enumerate}
\item Al servidor le llega una petición de registro por parte del cliente.
\item El servidor comprueba si los datos del usuario existen en la base de datos.
\item En caso de no existir los almacena en la base de datos y crea al jugador.
\item El servidor confirma al usuario que el jugador ha sido creado.
\end{enumerate} \\

\textbf{Poscondición} & Un nuevo usuario existe en el sistema.

\end{tabularx}
}

\subsection{Iniciar sesión}

{\footnotesize
\begin{tabularx}{0.95\textwidth}{p{0.2\textwidth}|X}

\textbf{Caso de uso} & Iniciar sesión \\

\textbf{Precondición} & Usuario registrado \\

\textbf{Escenario general} & \begin{enumerate}
\item Al servidor le llega una petición de inicio de sesión por parte del usuario.
\item El servidor comprueba la validez de los datos del usuario.
\item El servidor confirma al gestor de usuarios que el usuario es válido.
\item El gestor de usuarios crea una sesión.
\end{enumerate} \\

\textbf{Poscondición} & Se ejecuta el caso de uso ``Ver lista de partidas''
 y usuario autenticado\\ \\

\textbf{Escenario alternativo} & Puede ocurrir que los datos de acceso no sean
correctos, en cuyo caso se aborta la operación y se le comunica al usuario que los datos introducidos son erróneos.

\end{tabularx}
}

\subsection{Cerrar sesión}

{\footnotesize
\begin{tabularx}{0.95\textwidth}{p{0.2\textwidth}|X}

\textbf{Caso de uso} & Cerrar sesión \\

\textbf{Precondición} & Sesión iniciada \\

\textbf{Escenario general} & \begin{enumerate}
\item Al servidor le llega una petición de cierre de sesión por parte del usuario.
\item El servidor elmina al jugador de la lista de jugadores activos.
\item Si el jugador que cierra sesión está jugando una partida, se elminará al jugador de la partida.
\end{enumerate} \\

\textbf{Poscondición} & El usuario deja de estar activo en el servidor y borrado de la lista de jugadores.

\end{tabularx}
}

\subsection{Crear Partida}
{\footnotesize
\begin{tabularx}{0.95\textwidth}{p{0.2\textwidth}|X}

\textbf{Caso de uso} & Crear Partida \\

\textbf{Precondición} & Ambos usuarios deben estar logeados y uno debe haber retado a otro \\

\textbf{Escenario general} & \begin{enumerate}
\item El usuario selecciona de la ventana principal crear una nueva partida.
\item La ventana principal muestra la ventana de crear partida.
\item El usuario selecciona las características de la partida.
\item La ventana crear partida habilita el botón \emph{crear} cuando los datos
son correctos.
\item El usuario pulsa el botón crear.
\item La ventana crear partida notifica al gestor de partidas que un usuario
desea crear una partida.
\item El gestor de partidas pide al servidor mediante el proxy que cree una
partida.
\item El proxy le indica al gestor de partidas que se ha creado una nueva
partida.
\item El gestor modifica la ventana principal con el nuevo estado de las
partidas.
\item La ventana principal oculta la ventana de crear partidas.
\item La ventana principal comunica al jugador que se ha creado una nueva
partida.
\end{enumerate} \\

\textbf{Poscondición} & En el sistema hay una nueva partida y se ejecuta el caso
 de uso ``Unirse a partida''.\\ \\

\textbf{Escenario alternativo} & \begin{enumerate}
\item El usuario selecciona de la ventana principal crear una nueva partida.
\item La ventana principal muestra la ventana de crear partida.
\item El usuario selecciona las características de la partida.
\item La ventana crear partida habilita el botón \emph{crear} cuando los datos
son correctos.
\item El usuario pulsa el botón crear.
\item La ventana crear partida notifica al gestor de partidas que un usuario
desea crear una partida.
\item El gestor de partidas pide al servidor mediante el proxy que cree una
partida.
\item El proxy le indica al gestor de partidas que no se ha podido crear la
partida.
\item La ventana principal oculta la ventana de crear partidas.
\item La ventana principal comunica al jugador que no se ha podido crear la
partida.
\end{enumerate} \\

\end{tabularx}
}

\subsection{Elminar partida}
{\footnotesize
\begin{tabularx}{0.95\textwidth}{p{0.2\textwidth}|X}

\textbf{Caso de uso} & Eliminar partida \\

\textbf{Precondición} & El jugador debe haber iniciado sesión y haberse unido a
la partida.\\

\textbf{Escenario general} & \begin{enumerate}
\item Al servidor le llega una petición eliminar partida por parte del usuario.
\item El servidor elimina la partida.
\item EL servidor notifica a ambos jugadores la eliminación de la partida.

\end{enumerate} \\

\textbf{Poscondición}

\end{tabularx}
}

\subsection{Registrar Victorias}
{\footnotesize
\begin{tabularx}{0.95\textwidth}{p{0.2\textwidth}|X}

\textbf{Caso de uso} & Registrar Victorias \\

\textbf{Precondición} & Los usuarios deben estar conectados a una partida y la partida debe haber finalizado.\\

\textbf{Escenario general} & \begin{enumerate}
\item El servidor detecta que una partida ha finalizado y cúal es el jugador ganador.
\item El servidor almacena en la base de datos al ganador de la partida.
\item Notifica al los usuarios que la partida ha sido finalizada y qué jugador ha sido el vencedor.

\end{enumerate} \\

\textbf{Poscondición} & Se elimina la partida.

\end{tabularx}
}

\subsection{Ver la lista de partidas}

{\footnotesize
\begin{tabularx}{0.95\textwidth}{p{0.2\textwidth}|X}

\textbf{Caso de uso} & Ver la lista de partidas \\

\textbf{Precondición} & Iniciar sesión \\

\textbf{Escenario general} & \begin{enumerate}
\item La ventana principal pide al gestor de partidas una lista con las
partidas libres disponibles y una lista con las partidas que el usuario esta
jugando.
\item El gestor de partidas pide al servidor ambas listas.
\item El servidor envía ambas listas de partidas al gestor de partidas.
\item La ventana principal muestra la lista de partidas disponibles y la lista
de partidas comenzadas.

\end{enumerate} \\

\textbf{Poscondición} & La lista de partidas esta disponible para el usuario.

\end{tabularx}
}

\subsection{Enviar Petición de Reto}

{\footnotesize
\begin{tabularx}{0.95\textwidth}{p{0.2\textwidth}|X}

\textbf{Caso de uso} & Enviar petición de Reto\\

\textbf{Precondición} & Usuarios logeados en el sistea y petición de reto realizada. \\

\textbf{Escenario general} & \begin{enumerate}

\item Al servidor le llega una petición de reto por parte de un usuario.
\item El servidor envía al otro jugador la petición de reto.

\end{enumerate} \\

\textbf{Poscondición} &  \\ \\

\end{tabularx}
}

\subsection{Registrar un movimiento}

{\footnotesize
\begin{tabularx}{0.95\textwidth}{p{0.2\textwidth}|X}

\textbf{Caso de uso} & Registrar un movimiento \\

\textbf{Precondición} & Iniciar sesión, Unirse a una partida y Conectarse a una
partida \\

\textbf{Escenario general} & \begin{enumerate}
\item Al servidor le llega el movimiento realizado por un usuario.
\item El servidor almacena el movimiento en la base de datos.
\item El servidor notifica al otro usuario que ya puede realizar su movimiento.
\end{enumerate} \\

\textbf{Poscondición} & Se realiza el caso de uso: ''Enviar actualización de la
partida''.\\

\end{tabularx}
}

\subsection{Responder a solicitud de reto}

{\footnotesize
\begin{tabularx}{0.95\textwidth}{p{0.2\textwidth}|X}

\textbf{Caso de uso} & Responder a solicitud de reto \\

\textbf{Precondición} & Iniciar sesión\\

\textbf{Escenario general} & \begin{enumerate}
\item El servidor envía al usuario la solicitud de reto del otro usuario.
\item El cliente recibe la solicitud y la envía al controlador.
\item El controlador la envía a la ventana de jugadores para que muestre que el jugador ha sido retado
\item El usuario elige la acción en la ventana (si acepta o si no).
\item La ventana de juego envía la información al controlador.
\item La fachada envía la respuesta al servidor por medio del proxy.
\end{enumerate} \\

\textbf{Poscondición} & Respuesta a reto enviada.

\end{tabularx}
}

\subsection{Realizar movimiento}

{\footnotesize
\begin{tabularx}{0.95\textwidth}{p{0.2\textwidth}|X}

\textbf{Caso de uso} & Realizar Movimiento \\

\textbf{Precondición} & El usuario debe haberse unido a una partida y estar
conectado a ella. \\

\textbf{Escenario general} & \begin{enumerate}
\item El usuario pulsa sobre la casilla del tablero a colocar.
\item La interfaz avisa al controlador del movimiento.
\item El controlador envía el movimiento a la fachada.
\item La fachada envía el movimiento a Tablero9x9.
\item Tablero9x9 comprueba la legalidad del movimiento.
\item Si el movimiento es válido se lo envía al proxy y avisa al controlador para realizar el camino inverso hacia la interfaz y colocar la ficha.
\item Si el movimiento es inválido se lanza una excepción que se recoge en el controlador.
\item El controlador envía la excepción a la interfaz de usuario para comunicarle el error que ha cometido.
\end{enumerate} \\
\textbf{Postcondición} & Movimiento realizado. \\
\end{tabularx}
}

\subsection{Enviar petición de alianza}

{\footnotesize
\begin{tabularx}{0.95\textwidth}{p{0.2\textwidth}|X}

\textbf{Caso de uso} & Enviar petición de alianza \\

\textbf{Precondición} & El usuario debe haberse unido a una partida \\

\textbf{Escenario general} & \begin{enumerate}
\item El usuario pulsa sobre el botón alianza.
\item Se muestra al usuario una ventana con diferentes jugadores.
\item El usuario selecciona uno de ellos y confirma la operación.
\item El gestor del juego recibe los datos y envía la información al proxy.
\item El proxy notifica al gestor de juego que la alianza se ha realizado con
éxito.
\item El gestor de juego actualiza los datos del juego.
\end{enumerate} \\
\textbf{Postcondición} & Mostrar correctamente los datos de juego una vez
actualizado y se ejecuta el caso de uso  ``Enviar actualización de
la partida''. \\ \\
\textbf{Escenario alternativo} & \begin{enumerate}
\item El usuario pulsa sobre el botón alianza.
\item Se muestra al usuario una ventana con diferentes jugadores.
\item El usuario selecciona uno de ellos y confirma la operación.
\item El gestor del juego recibe los datos, comprueba que se cumple la condición
que permite la alianza, pero el jugador seleccionado no es válido para una
alianza.
\item El gestor de juego pide a la ventana que muestre un mensaje con
información sobre el error producido al realizar la alianza.
\end{enumerate}\\
\end{tabularx}
}

\subsection{Responder a petición de alianza}

{\footnotesize
\begin{tabularx}{0.95\textwidth}{p{0.2\textwidth}|X}
\textbf{Caso de uso} & Responder a petición de alianza \\

\textbf{Precondición} & El usuario debe haberse unido a una partida \\

\textbf{Escenario general} & \begin{enumerate}
\item El proxy comunica al gestor de juego que existe una petición de alianza.
\item El gestor de juego pide a la ventana que muestre al jugador el mensaje
pertinente.
\item El usuario selecciona la acción oportuna.
\item El gestor del juego recibe los datos y envía la información al proxy.
\item El proxy notifica al gestor de juego que la alianza se ha realizado con
éxito.
\item El gestor de juego actualiza los datos del juego.
\end{enumerate} \\
\textbf{Postcondición} & Mostrar correctamente los datos de juego una vez
actualizado Se ejecuta el caso de uso  ``Enviar actualización de
la partida''. .
\end{tabularx}
}

\subsection{Romper alianza}

{\footnotesize
\begin{tabularx}{0.95\textwidth}{p{0.2\textwidth}|X}

\textbf{Caso de uso} & Romper alianza \\

\textbf{Precondición} & El usuario debe haberse unido a una partida y tener, al
menos, una alianza \\

\textbf{Escenario general} & \begin{enumerate}
\item El usuario pulsa sobre el botón alianza.
\item Se muestra al usuario una ventana con diferentes alianzas.
\item El usuario selecciona una de ellas y confirma la operación.
\item El gestor del juego recibe los datos y envía la información al proxy.
\item El proxy notifica al gestor de juego que la alianza se ha roto con éxito.
\item El gestor de juego actualiza los datos del juego.
\end{enumerate} \\

\textbf{Postcondición} & Mostrar correctamente los datos de juego una vez
actualizado. Se ejecuta el caso de uso  ``Enviar actualización de
la partida''. .

\end{tabularx}
}

\subsection{Enviar actualización de la partida}

{\footnotesize
\begin{tabularx}{0.95\textwidth}{p{0.2\textwidth}|X}

\textbf{Caso de uso} & Enviar actualización de la partida \\

\textbf{Precondición} & El usuario debe haberse unido a una partida \\

\textbf{Escenario general} & \begin{enumerate}
\item El usuario realiza una operación que modifica la partida.
\item El gestor de juego recopila todos los datos modificados y envía la
información al proxy.
\item El proxy notifica al gestor de juego que la actualización se ha realizado
con éxito.
\end{enumerate} \\
\textbf{Postcondición} & Los datos del cliente y del servidor están
sincronizados.

\end{tabularx}
}

\subsection{Recibir actualización de la partida}

{\footnotesize
\begin{tabularx}{0.95\textwidth}{p{0.2\textwidth}|X}

\textbf{Caso de uso} & Recibir actualización de la partida \\

\textbf{Precondición} & El usuario debe haberse unido a una partida \\

\textbf{Escenario general} & \begin{enumerate}
\item El proxy comunica al gestor de juego que la partida ha cambiado.
\item El gestor de juego recibe todos los datos modificados y pide a la ventana
que se actualice.
\end{enumerate} \\
\textbf{Postcondición} & Los datos del cliente y del servidor están
sincronizados.

\end{tabularx}
}

% \section{Requisitos no funcionales}
%
% \subsection{Estructura lógica de los datos}
%
% \subsubsection{Entidad Datos de registro}
% \begin{tabularx}{0.9\textwidth}{llX}
% \hline
% \textbf{Elemento} & \textbf{Tipo} & \textbf{Descripción} \\
% \hline
% Nombre & Texto & Nombre del usuario \\
% eMail & Texto & Dirección de correo electrónico \\
% Contraseña & Texto & Contraseña en el sistema \\
% Conf. Contraseña & Texto & Confirmación de contraseña \\
% \hline
% \end{tabularx}



\section{Otros requisitos}
\subsection{Requisitos de la interfaz de usuario}
\subsubsection{Ventana de inicio}
{\footnotesize

 \textbf{Descripción:} Da la bienvenida al usuario.\\ Debe permitir al usuario
ejecutar el
caso de uso ``Iniciar sesión'' e introducir los siguientes datos: \\

\begin{tabularx}{0.9\textwidth}{llX}
\hline
\textbf{Dato} & \textbf{Tipo} & \textbf{Descripción} \\
\hline
NombreUsuario & Texto & Nombre del usuario \\
Contraseña & Texto & Contraseña en el sistema \\
\hline
\end{tabularx}

\subsubsection{Ventana de registro}
{\footnotesize



 \textbf{Descripción:} Permite registrarse al usuario.\\
Es necesario que
permita al usuario introducir los siguientes datos: \\

\begin{tabularx}{0.9\textwidth}{llX}
\hline
\textbf{Dato} & \textbf{Tipo} & \textbf{Descripción} \\
\hline
Nombre & Texto & Nombre del usuario \\
eMail & Texto & Dirección de correo electrónico \\
Contraseña & Texto & Contraseña en el sistema \\
Conf. Contraseña & Texto & Confirmación de contraseña \\
\hline
\end{tabularx}

}

\subsubsection{Ventana principal}
{\footnotesize

 \textbf{Descripción:}  Mostrará al usuario las partidas disponibles y las
partidas en
juego.\\
Es necesario que
permita al usuario ejecutar los siguientes casos de uso: \\
\begin{enumerate}
\item Crear una partida.
\item Unirse a una partida.
\item Conectarse a una partida.
\item Ver lista de partidas.

\end{enumerate}
}

\subsubsection{Ventana Crear partida}
{\footnotesize

 \textbf{Descripción:}  Permite al usuario crear una nueva partida.\\
Es necesario que
permita al usuario introducir los siguientes datos: \\

\begin{tabularx}{0.9\textwidth}{llX}
\hline
\textbf{Dato} & \textbf{Tipo} & \textbf{Descripción} \\
\hline
Nombre & Texto & Nombre de la partida \\
Días de juego & Fecha & Días en los que se jugará la partida\\
Hora de Inicio de Juego & Hora & Hora en la que empieza o continua la partida
los días indicados\\
Hora de Fin de Juego & Hora & Hora en la que termina la partida los días
indicados\\
\hline
\end{tabularx}
}

\subsubsection{Ventana de juego}
{\footnotesize



 \textbf{Descripción:}  Muestra al usuario todos los detalles de una partida y
permite
jugar en ella. Deberá mostrar el tablero de juego con la asignación de
territorios.\\

Es necesario que
permita al usuario ejecutar los siguientes casos de uso: \\
\begin{enumerate}
\item Desconectarse de una partida.
\item Realizar un movimiento.
\item Responder acción enemiga.
\item Comprar refuerzos.
\item Enviar petición de alianza.
\item Responder a petición de alianza.
\item Romper alianza.

\end{enumerate}


}
