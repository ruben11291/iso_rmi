\documentclass[a4paper,11pt,oneside]{article}
\usepackage[pdftex]{graphicx}
\usepackage[utf8]{inputenc}
\usepackage[spanish]{babel}
\usepackage{fancyhdr}
\usepackage[usenames,dvipsnames]{color}
\usepackage{colortbl}
\usepackage{tabularx}
\usepackage{hyperref}
\usepackage{pdfpages}


\setlength{\headheight}{15.2pt}
\pagestyle{fancy}

\lhead{\nouppercase{\leftmark}}
\chead{}
\rhead{}

\hypersetup{
colorlinks,
citecolor=black,
filecolor=black,
linkcolor=black,
urlcolor=black
}

\setlength{\parskip}{6pt}

\begin{document}

%%%% Title Page %%%%%

%\pagenumbering{alph}

\begin{titlepage}
\begin{center}

% Upper part of the page
\includegraphics[width=0.2\textwidth]{img/logo-uclm.png}\\[1cm]
\textsc{\LARGE Escuela Superior de Informática}\\[0.5cm]
\textsc{\Large Universidad de Castilla -- La Mancha}\\[2.5cm]

% Title
{\LARGE Práctica 1}\\[0.5cm]
%\rule{\linewidth}{0.5mm}\\[0.4cm]
{\huge \textbf{Tres en raya definitivo}}\\[0.4cm]
%\rule{\linewidth}{0.5mm}\\[1.5cm]

% Authors

  \begin{minipage}{0.8\textwidth}
  \large
  %\hspace{1cm}\textbf{\emph{Autores}}\\

  Javier Marchán Loro\\
  Ángel Peralta López\\
  Rubén Pérez Pascual\\
  Antonio Ruedas García\\
  \end{minipage}

  \vfill

% Bottom of the page
\begin{minipage}{\textwidth}
\large
\begin{tabular}{rl}
\textbf{Asignatura}: & Ingeniería del Software II\\
\textbf{Titulación}: & Ingeniería Informática\\
\textbf{Fecha}: & \today
\end{tabular}
\end{minipage}

\end{center}
\end{titlepage}

%%%% end Title Page %%%%%

\clearpage
\section{Introducción}

Este documento describe el análisis, desarrollo e implementación llevados a cabo para obtener
el juego \textbf{Tres en raya definitivo} \footnote{Este documento no explica las reglas del juego}.

Se trata de una aplicación cliente-servidor para escritorio que se comunica a través de RMI.
El lenguaje utilizado para la implementación ha sido Java.

Este documento pretende que el lector obtenga una idea clara de la composición del sistema, para ello
se ha dividido en distintas secciones. En primer lugar los requisitos funcionales ayudan a entender el objetivo
buscado. La arquitectura del sistema junto con los casos de uso son las secciones más importantes, ya
que ayudarán al lector a hacerse una imagen mental, a alto nivel, del sistema desarrollado.

El resto de secciones profundizan en el conocimiento del sistema.

\clearpage

\pagestyle{fancy}

\section{Requisitos funcionales}
 % \section{Requisitos de interfaces externas}

% \section{Requisitos funcionales}

\subsection{Registrarse}

{\footnotesize
\begin{tabularx}{0.95\textwidth}{p{0.2\textwidth}|X}

\textbf{Caso de uso} & Registrarse \\

\textbf{Precondición} & Ninguna \\

\textbf{Escenario general} & \begin{enumerate}
\item El usuario pulsa el botón de registrarse en la ventana de inicio.
\item Se muestra la ventana de registro.
\item El usuario introduce su email y su contraseña en la ventana de registro.
\item El usuario pulsa el botón de confirmación.
\item Se espera confirmación por parte del servidor.
\item La ventana de inicio informa al usuario del éxito de la operación.
\end{enumerate} \\

\textbf{Poscondición} & Un nuevo usuario existe en el sistema.

\end{tabularx}
}

\subsection{Iniciar sesión}

{\footnotesize
\begin{tabularx}{0.95\textwidth}{p{0.2\textwidth}|X}

\textbf{Caso de uso} & Iniciar sesión \\

\textbf{Precondición} & Usuario registrado \\

\textbf{Escenario general} & \begin{enumerate}
\item El usuario introduce su email y contraseña en la ventana de inicio.
\item El usuario pulsa el botón de ``Iniciar sesión''.
\item Se espera validación por parte del servidor.
\item Se cierra la ventana de login y se abre la lista de jugadores.
\end{enumerate} \\

\textbf{Poscondición} & Se ejecuta el caso de uso ``Ver lista de jugadores''
\\ \\

\textbf{Escenario alternativo} & Puede ocurrir que los datos de acceso no sean
correctos, en cuyo caso se aborta la operación y se vuelve al estado inicial.

\end{tabularx}
}

\subsection{Cerrar sesión}

{\footnotesize
\begin{tabularx}{0.95\textwidth}{p{0.2\textwidth}|X}

\textbf{Caso de uso} & Cerrar sesión \\

\textbf{Precondición} & Sesión iniciada \\

\textbf{Escenario general} & \begin{enumerate}
\item El usuario pulsa el botón de cerrar el programa.
\item Se manda una petición de cierre de sesión al servidor.
\item El servidor cierra la sesión del usuario.
\item Se cierra el programa.
\end{enumerate} \\

\textbf{Poscondición} & El usuario deja de estar activo en el servidor y deja de aparecer
en la lista de jugadores.

\textbf{Precondición} & Sesión iniciada y jugador en partida \\

\textbf{Escenario alternativo} & \begin{enumerate}
\item El usuario pulsa el botón de cerrar el programa.
\item Se manda una petición de cierre de sesión al servidor.
\item El servidor elimina la partida y cierra la sesión del usuario.
\item Se cierra el programa.
\end{enumerate} \\

\textbf{Poscondición} & El usuario deja de estar activo en el servidor y deja de aparecer
en la lista de jugadores.

\end{tabularx}
}

\subsection{Salir de la partida}
{\footnotesize
\begin{tabularx}{0.95\textwidth}{p{0.2\textwidth}|X}

\textbf{Caso de uso} & Salir de la partida \\

\textbf{Precondición} & Estar conectado a una partida.\\

\textbf{Escenario general} & \begin{enumerate}
\item El usuario pulsa el botón cerrar partida en la ventana de juego.
\item Se comunica al servidor que elimmine la partida.
\item Se cierra la ventana de juego.

\end{enumerate} \\

\textbf{Poscondición} & Se sale de la partida y aparece la ventana principal.

\end{tabularx}
}

\subsection{Retar jugador}
{\footnotesize
\begin{tabularx}{0.95\textwidth}{p{0.2\textwidth}|X}

\textbf{Caso de uso} & Retar jugador \\

\textbf{Precondición} & Sesión iniciada \\

\textbf{Escenario general} & \begin{enumerate}
\item El usuario selecciona un jugador de la lista y pulsa ``Retar''.
\item Se manda la petición al servidor que a su vez se lo comunica al oponente.

\end{enumerate} \\

\textbf{Poscondición} & Usuario espera confirmación del oponente.

\end{tabularx}
}

\subsection{Aceptar reto}
{\footnotesize
\begin{tabularx}{0.95\textwidth}{p{0.2\textwidth}|X}

\textbf{Caso de uso} & Aceptar reto \\

\textbf{Precondición} & Sesión iniciada \\

\textbf{Escenario general} & \begin{enumerate}
\item Se abre ventana emergente de invitación a partida.
\item El usuario pulsa el botón de aceptar.

\end{enumerate} \\

\textbf{Poscondición} & Se crea una partida.

\end{tabularx}
}

\subsection{Realizar movimiento}
{\footnotesize
\begin{tabularx}{0.95\textwidth}{p{0.2\textwidth}|X}

\textbf{Caso de uso} & Realizar movimiento \\

\textbf{Precondición} & En una partida \\

\textbf{Escenario general} & \begin{enumerate}
\item El usuario coloca una pieza en el tablero de juego.
\item Se comprueba la validez del movimiento.
\item Se manda información del movimiento al servidor.

\end{enumerate} \\

\textbf{Poscondición} & Tablero modificado.

\textbf{Escenario general} & \begin{enumerate}
\item El usuario coloca una pieza en el tablero de juego.
\item Se comprueba la validez del movimiento.
\item Se informa al usuario de que el movimiento no es válido.

\end{enumerate} \\

\textbf{Poscondición} & Tablero no modificado.

\end{tabularx}
}

\subsection{Ver la lista de jugadores}

{\footnotesize
\begin{tabularx}{0.95\textwidth}{p{0.2\textwidth}|X}

\textbf{Caso de uso} & Ver la lista de jugadores \\

\textbf{Precondición} & Iniciar sesión \\

\textbf{Escenario general} & \begin{enumerate}
\item Se abre la ventana de la lista de jugadores y recibe la lista de
usuarios conectados del servidor.

\end{enumerate} \\

\textbf{Poscondición} & La lista de jugadores está disponible para el usuario.

\end{tabularx}
}

\end{enumerate}


}

%
\subsection{Crear Partida}
{\footnotesize
\begin{tabularx}{0.95\textwidth}{p{0.2\textwidth}|X}

\textbf{Caso de uso} & Crear Partida \\

\textbf{Precondición} & Ambos usuarios deben estar logeados y uno debe haber retado a otro \\

\textbf{Escenario general} & \begin{enumerate}
\item El usuario selecciona de la ventana principal crear una nueva partida.
\item La ventana principal muestra la ventana de crear partida.
\item El usuario selecciona las características de la partida.
\item La ventana crear partida habilita el botón \emph{crear} cuando los datos
son correctos.
\item El usuario pulsa el botón crear.
\item La ventana crear partida notifica al gestor de partidas que un usuario
desea crear una partida.
\item El gestor de partidas pide al servidor mediante el proxy que cree una
partida.
\item El proxy le indica al gestor de partidas que se ha creado una nueva
partida.
\item El gestor modifica la ventana principal con el nuevo estado de las
partidas.
\item La ventana principal oculta la ventana de crear partidas.
\item La ventana principal comunica al jugador que se ha creado una nueva
partida.
\end{enumerate} \\

\textbf{Poscondición} & En el sistema hay una nueva partida y se ejecuta el caso
 de uso ``Unirse a partida''.\\ \\

\textbf{Escenario alternativo} & \begin{enumerate}
\item El usuario selecciona de la ventana principal crear una nueva partida.
\item La ventana principal muestra la ventana de crear partida.
\item El usuario selecciona las características de la partida.
\item La ventana crear partida habilita el botón \emph{crear} cuando los datos
son correctos.
\item El usuario pulsa el botón crear.
\item La ventana crear partida notifica al gestor de partidas que un usuario
desea crear una partida.
\item El gestor de partidas pide al servidor mediante el proxy que cree una
partida.
\item El proxy le indica al gestor de partidas que no se ha podido crear la
partida.
\item La ventana principal oculta la ventana de crear partidas.
\item La ventana principal comunica al jugador que no se ha podido crear la
partida.
\end{enumerate} \\

\end{tabularx}
}





\subsection{Ver la lista de partidas}

{\footnotesize
\begin{tabularx}{0.95\textwidth}{p{0.2\textwidth}|X}

\textbf{Caso de uso} & Ver la lista de partidas \\

\textbf{Precondición} & Iniciar sesión \\

\textbf{Escenario general} & \begin{enumerate}
\item La ventana principal pide al gestor de partidas una lista con las
partidas libres disponibles y una lista con las partidas que el usuario esta
jugando.
\item El gestor de partidas pide al servidor ambas listas.
\item El servidor envía ambas listas de partidas al gestor de partidas.
\item La ventana principal muestra la lista de partidas disponibles y la lista
de partidas comenzadas.

\end{enumerate} \\

\textbf{Poscondición} & La lista de partidas esta disponible para el usuario.

\end{tabularx}
}


\subsection{Enviar petición de alianza}

{\footnotesize
\begin{tabularx}{0.95\textwidth}{p{0.2\textwidth}|X}

\textbf{Caso de uso} & Enviar petición de alianza \\

\textbf{Precondición} & El usuario debe haberse unido a una partida \\

\textbf{Escenario general} & \begin{enumerate}
\item El usuario pulsa sobre el botón alianza.
\item Se muestra al usuario una ventana con diferentes jugadores.
\item El usuario selecciona uno de ellos y confirma la operación.
\item El gestor del juego recibe los datos y envía la información al proxy.
\item El proxy notifica al gestor de juego que la alianza se ha realizado con
éxito.
\item El gestor de juego actualiza los datos del juego.
\end{enumerate} \\
\textbf{Postcondición} & Mostrar correctamente los datos de juego una vez
actualizado y se ejecuta el caso de uso  ``Enviar actualización de
la partida''. \\ \\
\textbf{Escenario alternativo} & \begin{enumerate}
\item El usuario pulsa sobre el botón alianza.
\item Se muestra al usuario una ventana con diferentes jugadores.
\item El usuario selecciona uno de ellos y confirma la operación.
\item El gestor del juego recibe los datos, comprueba que se cumple la condición
que permite la alianza, pero el jugador seleccionado no es válido para una
alianza.
\item El gestor de juego pide a la ventana que muestre un mensaje con
información sobre el error producido al realizar la alianza.
\end{enumerate}\\
\end{tabularx}
}

\subsection{Responder a petición de alianza}

{\footnotesize
\begin{tabularx}{0.95\textwidth}{p{0.2\textwidth}|X}
\textbf{Caso de uso} & Responder a petición de alianza \\

\textbf{Precondición} & El usuario debe haberse unido a una partida \\

\textbf{Escenario general} & \begin{enumerate}
\item El proxy comunica al gestor de juego que existe una petición de alianza.
\item El gestor de juego pide a la ventana que muestre al jugador el mensaje
pertinente.
\item El usuario selecciona la acción oportuna.
\item El gestor del juego recibe los datos y envía la información al proxy.
\item El proxy notifica al gestor de juego que la alianza se ha realizado con
éxito.
\item El gestor de juego actualiza los datos del juego.
\end{enumerate} \\
\textbf{Postcondición} & Mostrar correctamente los datos de juego una vez
actualizado Se ejecuta el caso de uso  ``Enviar actualización de
la partida''. .
\end{tabularx}
}

\subsection{Romper alianza}

{\footnotesize
\begin{tabularx}{0.95\textwidth}{p{0.2\textwidth}|X}

\textbf{Caso de uso} & Romper alianza \\

\textbf{Precondición} & El usuario debe haberse unido a una partida y tener, al
menos, una alianza \\

\textbf{Escenario general} & \begin{enumerate}
\item El usuario pulsa sobre el botón alianza.
\item Se muestra al usuario una ventana con diferentes alianzas.
\item El usuario selecciona una de ellas y confirma la operación.
\item El gestor del juego recibe los datos y envía la información al proxy.
\item El proxy notifica al gestor de juego que la alianza se ha roto con éxito.
\item El gestor de juego actualiza los datos del juego.
\end{enumerate} \\

\textbf{Postcondición} & Mostrar correctamente los datos de juego una vez
actualizado. Se ejecuta el caso de uso  ``Enviar actualización de
la partida''. .

\end{tabularx}
}

\subsection{Enviar actualización de la partida}

{\footnotesize
\begin{tabularx}{0.95\textwidth}{p{0.2\textwidth}|X}

\textbf{Caso de uso} & Enviar actualización de la partida \\

\textbf{Precondición} & El usuario debe haberse unido a una partida \\

\textbf{Escenario general} & \begin{enumerate}
\item El usuario realiza una operación que modifica la partida.
\item El gestor de juego recopila todos los datos modificados y envía la
información al proxy.
\item El proxy notifica al gestor de juego que la actualización se ha realizado
con éxito.
\end{enumerate} \\
\textbf{Postcondición} & Los datos del cliente y del servidor están
sincronizados.

\end{tabularx}
}

\subsection{Recibir actualización de la partida}

{\footnotesize
\begin{tabularx}{0.95\textwidth}{p{0.2\textwidth}|X}

\textbf{Caso de uso} & Recibir actualización de la partida \\

\textbf{Precondición} & El usuario debe haberse unido a una partida \\

\textbf{Escenario general} & \begin{enumerate}
\item El proxy comunica al gestor de juego que la partida ha cambiado.
\item El gestor de juego recibe todos los datos modificados y pide a la ventana
que se actualice.
\end{enumerate} \\
\textbf{Postcondición} & Los datos del cliente y del servidor están
sincronizados.

\end{tabularx}
}

% \section{Requisitos no funcionales}
%
% \subsection{Estructura lógica de los datos}
%
% \subsubsection{Entidad Datos de registro}
% \begin{tabularx}{0.9\textwidth}{llX}
% \hline
% \textbf{Elemento} & \textbf{Tipo} & \textbf{Descripción} \\
% \hline
% Nombre & Texto & Nombre del usuario \\
% eMail & Texto & Dirección de correo electrónico \\
% Contraseña & Texto & Contraseña en el sistema \\
% Conf. Contraseña & Texto & Confirmación de contraseña \\
% \hline
% \end{tabularx}



\section{Otros requisitos}
\subsection{Requisitos de la interfaz de usuario}
\subsubsection{Ventana de inicio}
{\footnotesize

 \textbf{Descripción:} Da la bienvenida al usuario.\\ Debe permitir al usuario
ejecutar el
caso de uso ``Iniciar sesión'' e introducir los siguientes datos: \\

\begin{tabularx}{0.9\textwidth}{llX}
\hline
\textbf{Dato} & \textbf{Tipo} & \textbf{Descripción} \\
\hline
NombreUsuario & Texto & Nombre del usuario \\
Contraseña & Texto & Contraseña en el sistema \\
\hline
\end{tabularx}

\subsubsection{Ventana de registro}
{\footnotesize



 \textbf{Descripción:} Permite registrarse al usuario.\\
Es necesario que
permita al usuario introducir los siguientes datos: \\

\begin{tabularx}{0.9\textwidth}{llX}
\hline
\textbf{Dato} & \textbf{Tipo} & \textbf{Descripción} \\
\hline
Nombre & Texto & Nombre del usuario \\
eMail & Texto & Dirección de correo electrónico \\
Contraseña & Texto & Contraseña en el sistema \\
Conf. Contraseña & Texto & Confirmación de contraseña \\
\hline
\end{tabularx}

}

\subsubsection{Ventana principal}
{\footnotesize

 \textbf{Descripción:}  Mostrará al usuario las partidas disponibles y las
partidas en
juego.\\
Es necesario que
permita al usuario ejecutar los siguientes casos de uso: \\
\begin{enumerate}
\item Crear una partida.
\item Unirse a una partida.
\item Conectarse a una partida.
\item Ver lista de partidas.

\end{enumerate}
}

\subsubsection{Ventana Crear partida}
{\footnotesize

 \textbf{Descripción:}  Permite al usuario crear una nueva partida.\\
Es necesario que
permita al usuario introducir los siguientes datos: \\

\begin{tabularx}{0.9\textwidth}{llX}
\hline
\textbf{Dato} & \textbf{Tipo} & \textbf{Descripción} \\
\hline
Nombre & Texto & Nombre de la partida \\
Días de juego & Fecha & Días en los que se jugará la partida\\
Hora de Inicio de Juego & Hora & Hora en la que empieza o continua la partida
los días indicados\\
Hora de Fin de Juego & Hora & Hora en la que termina la partida los días
indicados\\
\hline
\end{tabularx}
}

\subsubsection{Ventana de juego}
{\footnotesize



 \textbf{Descripción:}  Muestra al usuario todos los detalles de una partida y
permite
jugar en ella. Deberá mostrar el tablero de juego con la asignación de
territorios.\\

Es necesario que
permita al usuario ejecutar los siguientes casos de uso: \\
\begin{enumerate}
\item Desconectarse de una partida.
\item Realizar un movimiento.
\item Responder acción enemiga.
\item Comprar refuerzos.
\item Enviar petición de alianza.
\item Responder a petición de alianza.
\item Romper alianza.

\end{enumerate}


}

\clearpage
\fancyhf{}
\includepdf[pages=4-23, pagecommand={}]{quienloborreloMato.pdf}
\setcounter{page}{\value{page}}
\clearpage
%\pagestyle{fancy}
\lhead{\nouppercase{\leftmark}}
\chead{}
\rhead{}
\setcounter{section}{7}
\section{Pruebas de los casos de uso}

\subsection{Pruebas para el servidor}

Las pruebas para el sistema del servidor se han hecho de tal forma que puedan ser ejecutadas de forma independiente. No necesita ningún cliente. El paquete test.communications contiene un cliente construido específicamente para las pruebas y consiste en un modelo simplificado de la clase Cliente del sistema cliente.

\begin{itemize}
\item Paquete \emph{src.server.tests}

\begin{center}
{\footnotesize
\begin{tabularx}{0.95\textwidth}{p{0.2\textwidth}|X}

\textbf{Caso de Uso} & \textbf{Test} \\
& \\
Registrar Usuario & TestDAOJugador\\
Autenticar & TestDAOJugador\\
Cerrar Sesión & TestCerrarSesion\\
Registrar Moviminento & TestDAOMovimiento\\
Enviar movimiento a jugador & TestReto\\
Enviar respuesta a reto & TestReto\\
\end{tabularx}
}
\end{center}

\item Paquete \emph{src.server.tests.communications}.
\begin{center}
{\footnotesize
\begin{tabularx}{0.95\textwidth}{p{0.2\textwidth}|X}
\textbf{Caso de Uso} & \textbf{Test} \\
& \\
Registrar Usuario & TestDAOJugador\\
Autenticar & TestDAOJugador\\
Cerrar Sesión & TestCerrarSesion\\
Registrar Moviminento & TestDAOMovimiento\\
Enviar movimiento a jugador & TestReto\\
Enviar respuesta a reto & TestReto\\
\end{tabularx}
}
\end{center}
\end{itemize}

Los test no prueban la comunicación entre clientes, solo la funcionalidad del servidor a partir de la fachada del servidor (clase FTERD, ver Diagrama de clases del servidor).
\clearpage

\subsection{Pruebas para el cliente}


% \pagenumbering{arabic}

%\includepdf[pages=4-, pagecommand={\thispagestyle{empty}}]{memoria_finalJavi.pdf}
%\includepdf[pages=4-, pagecommand={\setcounter{page}{32}}]{memoria_finalJavi.pdf}
%\includepdf[pages=4-, pagecommand={}]{quienloborreloMato.pdf}
%\setcounter{page}{\value{page}}
%\thispagestyle{empty} % no page number
\end{document}
