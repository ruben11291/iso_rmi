% \section{Requisitos de interfaces externas}

% \section{Requisitos funcionales}

\subsection{Registrarse}

{\footnotesize
\begin{tabularx}{0.95\textwidth}{p{0.2\textwidth}|X}

\textbf{Caso de uso} & Registrarse \\

\textbf{Precondición} & Ninguna \\

\textbf{Escenario general} & \begin{enumerate}
\item El usuario pulsa el botón de registrarse en la ventana de inicio.
\item Se muestra la ventana de registro.
\item El usuario introduce su email y su contraseña en la ventana de registro.
\item El usuario pulsa el botón de confirmación.
\item Se espera confirmación por parte del servidor.
\item La ventana de inicio informa al usuario del éxito de la operación.
\end{enumerate} \\

\textbf{Poscondición} & Un nuevo usuario existe en el sistema.

\end{tabularx}
}

\subsection{Iniciar sesión}

{\footnotesize
\begin{tabularx}{0.95\textwidth}{p{0.2\textwidth}|X}

\textbf{Caso de uso} & Iniciar sesión \\

\textbf{Precondición} & Usuario registrado \\

\textbf{Escenario general} & \begin{enumerate}
\item El usuario introduce su email y contraseña en la ventana de inicio.
\item El usuario pulsa el botón de ``Iniciar sesión''.
\item Se espera validación por parte del servidor.
\item Se cierra la ventana de login y se abre la lista de jugadores.
\end{enumerate} \\

\textbf{Poscondición} & Se ejecuta el caso de uso ``Ver lista de jugadores''
\\ \\

\textbf{Escenario alternativo} & Puede ocurrir que los datos de acceso no sean
correctos, en cuyo caso se aborta la operación y se vuelve al estado inicial.

\end{tabularx}
}

\subsection{Cerrar sesión}

{\footnotesize
\begin{tabularx}{0.95\textwidth}{p{0.2\textwidth}|X}

\textbf{Caso de uso} & Cerrar sesión \\

\textbf{Precondición} & Sesión iniciada \\

\textbf{Escenario general} & \begin{enumerate}
\item El usuario pulsa el botón de cerrar el programa.
\item Se manda una petición de cierre de sesión al servidor.
\item El servidor cierra la sesión del usuario.
\item Se cierra el programa.
\end{enumerate} \\

\textbf{Poscondición} & El usuario deja de estar activo en el servidor y deja de aparecer
en la lista de jugadores.

\textbf{Precondición} & Sesión iniciada y jugador en partida \\

\textbf{Escenario alternativo} & \begin{enumerate}
\item El usuario pulsa el botón de cerrar el programa.
\item Se manda una petición de cierre de sesión al servidor.
\item El servidor elimina la partida y cierra la sesión del usuario.
\item Se cierra el programa.
\end{enumerate} \\

\textbf{Poscondición} & El usuario deja de estar activo en el servidor y deja de aparecer
en la lista de jugadores.

\end{tabularx}
}

\subsection{Salir de la partida}
{\footnotesize
\begin{tabularx}{0.95\textwidth}{p{0.2\textwidth}|X}

\textbf{Caso de uso} & Salir de la partida \\

\textbf{Precondición} & Estar conectado a una partida.\\

\textbf{Escenario general} & \begin{enumerate}
\item El usuario pulsa el botón cerrar partida en la ventana de juego.
\item Se comunica al servidor que elimmine la partida.
\item Se cierra la ventana de juego.

\end{enumerate} \\

\textbf{Poscondición} & Se sale de la partida y aparece la ventana principal.

\end{tabularx}
}

\subsection{Retar jugador}
{\footnotesize
\begin{tabularx}{0.95\textwidth}{p{0.2\textwidth}|X}

\textbf{Caso de uso} & Retar jugador \\

\textbf{Precondición} & Sesión iniciada \\

\textbf{Escenario general} & \begin{enumerate}
\item El usuario selecciona un jugador de la lista y pulsa ``Retar''.
\item Se manda la petición al servidor que a su vez se lo comunica al oponente.

\end{enumerate} \\

\textbf{Poscondición} & Usuario espera confirmación del oponente.

\end{tabularx}
}

\subsection{Aceptar reto}
{\footnotesize
\begin{tabularx}{0.95\textwidth}{p{0.2\textwidth}|X}

\textbf{Caso de uso} & Aceptar reto \\

\textbf{Precondición} & Sesión iniciada \\

\textbf{Escenario general} & \begin{enumerate}
\item Se abre ventana emergente de invitación a partida.
\item El usuario pulsa el botón de aceptar.

\end{enumerate} \\

\textbf{Poscondición} & Se crea una partida.

\end{tabularx}
}

\subsection{Realizar movimiento}
{\footnotesize
\begin{tabularx}{0.95\textwidth}{p{0.2\textwidth}|X}

\textbf{Caso de uso} & Realizar movimiento \\

\textbf{Precondición} & En una partida \\

\textbf{Escenario general} & \begin{enumerate}
\item El usuario coloca una pieza en el tablero de juego.
\item Se comprueba la validez del movimiento.
\item Se manda información del movimiento al servidor.

\end{enumerate} \\

\textbf{Poscondición} & Tablero modificado.

\textbf{Escenario general} & \begin{enumerate}
\item El usuario coloca una pieza en el tablero de juego.
\item Se comprueba la validez del movimiento.
\item Se informa al usuario de que el movimiento no es válido.

\end{enumerate} \\

\textbf{Poscondición} & Tablero no modificado.

\end{tabularx}
}

\subsection{Ver la lista de jugadores}

{\footnotesize
\begin{tabularx}{0.95\textwidth}{p{0.2\textwidth}|X}

\textbf{Caso de uso} & Ver la lista de jugadores \\

\textbf{Precondición} & Iniciar sesión \\

\textbf{Escenario general} & \begin{enumerate}
\item Se abre la ventana de la lista de jugadores y recibe la lista de
usuarios conectados del servidor.

\end{enumerate} \\

\textbf{Poscondición} & La lista de jugadores está disponible para el usuario.

\end{tabularx}
}

\end{enumerate}


}
