\section{Requisitos funcionales}

\subsection{Registrarse}

{\footnotesize
\begin{tabularx}{0.95\textwidth}{p{0.2\textwidth}|X}

\textbf{Caso de uso} & \textbf{Registrarse} \\
& \\
\textbf{Precondición} & No está registrado \\

\textbf{Escenario general} & \begin{enumerate}
\item El usuario pulsa el botón de registrarse en la ventana de inicio.
\item Se muestra la ventana de registro.
\item El usuario introduce su email y su contraseña en la ventana de registro.
\item El usuario pulsa el botón de confirmación.
\item Se espera confirmación por parte del servidor.
\item La ventana de inicio informa al usuario del éxito de la operación.
\end{enumerate} \\

\textbf{Poscondición} & Un nuevo usuario existe en el sistema.\\
& \\
& \\
\textbf{Precondición} & Ya está registrado\\
\textbf{Escenario alternativo} & \begin{enumerate}
 \item El usuario pulsa el botón de registrarse en la ventana de inicio.
\item Se muestra la ventana de registro.
\item El usuario introduce su email y contraseña en la ventana de registro.
\item El usuario pulsa el botón de confirmación.
\item Se espera confirmación por parte del servidor.
\item La ventana de inicio informa al usuario que ya se encuenta registrado.
\end{enumerate}\\
\textbf{Poscondición}& No se registra el usuario de nuevo.\\

\end{tabularx}
}

\subsection{Iniciar sesión}

{\footnotesize
\begin{tabularx}{0.95\textwidth}{p{0.2\textwidth}|X}

\textbf{Caso de uso} & \textbf{Iniciar sesión} \\
& \\
\textbf{Precondición} & Usuario registrado y aún no logeado \\

\textbf{Escenario general} & \begin{enumerate}
\item El usuario introduce su email y contraseña en la ventana de inicio.
\item El usuario pulsa el botón de ``Iniciar sesión''.
\item Se espera validación por parte del servidor.
\item Se cierra la ventana de login y se abre la lista de jugadores.
\end{enumerate} \\

\textbf{Poscondición} & Se ejecuta el caso de uso ``Ver lista de jugadores''
\\
& \\
& \\
\textbf{Precondición} & Datos de acceso incorrectos\\
\textbf{Escenario alternativo} & \begin{enumerate}
\item El usuario introduce su email y contraseña en la ventana de inicio.
\item El usuario pulsa el botón de ``Iniciar sesión''.
\item Se espera validación por parte del servidor.
\item Se informa al usuario que los datos de acceso son incorrectos.
\end{enumerate}\\
\textbf{Poscondición} &Usuario no logeado.\\
& \\
& \\
\textbf{Precondición} & Usuario ya logeado.\\
\textbf{Escenario alternativo}& \begin {enumerate}
\item El usuario introduce su email y contraseña en la ventana de inicio.
\item El usuario pulsa el botón de ``Iniciar sesión''.
\item Se espera validación por parte del servidor.
\item Se informa al usuario que ya está logeado.
\end{enumerate}\\
\textbf{Poscondición}& Usuario no logeado de nuevo.\\
\end{tabularx}
}

\subsection{Cerrar sesión}

{\footnotesize
\begin{tabularx}{0.95\textwidth}{p{0.2\textwidth}|X}

\textbf{Caso de uso} & \textbf{Cerrar sesión} \\
& \\
\textbf{Precondición} & Sesión iniciada \\

\textbf{Escenario general} & \begin{enumerate}
\item El usuario pulsa el botón de cerrar el programa.
\item Se manda una petición de cierre de sesión al servidor.
\item El servidor cierra la sesión del usuario.
\item Se cierra el programa.
\end{enumerate} \\

\textbf{Poscondición} & El usuario deja de estar activo en el servidor y deja de aparecer
en la lista de jugadores.\\

& \\
& \\
\textbf{Precondición} & Sesión iniciada y jugador en partida \\

\textbf{Escenario alternativo} & \begin{enumerate}
\item El usuario pulsa el botón de cerrar el programa.
\item Se manda una petición de cierre de sesión al servidor.
\item El servidor elimina la partida y cierra la sesión del usuario.
\item Se cierra el programa.
\end{enumerate} \\

\textbf{Poscondición} & El usuario deja de estar activo en el servidor y deja de aparecer
en la lista de jugadores.

\end{tabularx}
}

\subsection{Salir de la partida}
{\footnotesize
\begin{tabularx}{0.95\textwidth}{p{0.2\textwidth}|X}

\textbf{Caso de uso} &\textbf{Salir de la partida} \\
& \\
\textbf{Precondición} & Estar conectado a una partida.\\

\textbf{Escenario general} & \begin{enumerate}
\item El usuario pulsa el botón cerrar partida en la ventana de juego.
\item Se comunica al servidor que elimmine la partida.
\item Se cierra la ventana de juego.

\end{enumerate} \\

\textbf{Poscondición} & Se sale de la partida y aparece la ventana principal.

\end{tabularx}
}

\subsection{Retar jugador}
{\footnotesize
\begin{tabularx}{0.95\textwidth}{p{0.2\textwidth}|X}

\textbf{Caso de uso} & \textbf{Retar jugador} \\
& \\
\textbf{Precondición} & Sesión iniciada y oponente en el sistema \\

\textbf{Escenario general} & \begin{enumerate}
\item El usuario selecciona un jugador de la lista y pulsa ``Retar''.
\item Se manda la petición al servidor que a su vez se lo comunica al oponente.
\item El servidor busca al oponente y le envia la solicitud de reto.
\end{enumerate} \\

\textbf{Poscondición} & Usuario espera confirmación del oponente.

\end{tabularx}
}

\subsection{Aceptar reto}
{\footnotesize
\begin{tabularx}{0.95\textwidth}{p{0.2\textwidth}|X}

\textbf{Caso de uso} & \textbf{Aceptar reto} \\
& \\
\textbf{Precondición} & Sesión iniciada y petición de reto realizada\\

\textbf{Escenario general} & \begin{enumerate}
\item Se abre ventana emergente de invitación a partida.
\item El usuario pulsa el botón de aceptar.
\item Se notifica al servidor de la aceptación.
\end{enumerate} \\

\textbf{Poscondición} & Se crea una partida.\\
& \\
& \\
\textbf{Precondición} & Sesión iniciada y petición de reto realizada\\
\textbf{Escenario alternativo} & \begin{enumerate}
\item Se abre ventana emergente de invitación a partida.
\item El usuario pulsa el botón de cancelar.
\item Se notifica al servidor de la cancelación.
\end{enumerate}\\

\textbf{Poscondición}& No se crea partida y se vuelve a la GUI de la lista de jugadores.

\end{tabularx}
}

\subsection{Realizar movimiento}
{\footnotesize
\begin{tabularx}{0.95\textwidth}{p{0.2\textwidth}|X}

\textbf{Caso de uso} & \textbf {Realizar movimiento válido} \\
& \\
\textbf{Precondición} & Existencia de una partida \\

\textbf{Escenario general} & \begin{enumerate}
\item El usuario coloca una pieza en el tablero de juego.
\item Se comprueba la validez del movimiento.
\item Se manda información del movimiento al servidor.

\end{enumerate} \\

\textbf{Poscondición} & Tablero modificado.\\
& \\
& \\
\textbf{Precondición} & Realizar movimiento no válido\\
\textbf{Escenario alternativo} & \begin{enumerate}
\item El usuario coloca una pieza en el tablero de juego.
\item Se comprueba la validez del movimiento.
\item Se informa al usuario de que el movimiento no es válido.

\end{enumerate} \\

\textbf{Poscondición} & Tablero no modificado.

\end{tabularx}
}

\subsection{Ver la lista de jugadores}

{\footnotesize
\begin{tabularx}{0.95\textwidth}{p{0.2\textwidth}|X}

\textbf{Caso de uso} & \textbf{Ver la lista de jugadores} \\
& \\
\textbf{Precondición} & Iniciar sesión \\

\textbf{Escenario general} & \begin{enumerate}
\item Se abre la ventana de la lista de jugadores y recibe la lista de
usuarios conectados del servidor.

\end{enumerate} \\

\textbf{Poscondición} & La lista de jugadores está disponible para el usuario.\\
 & \\
& \\
\textbf{Precondición} & Jugador ya iniciado sesión y otro jugador se conecta.\\
\textbf{Escenario alternativo} & \begin{enumerate}
\item El servidor envía al usuario la lista de jugadores actualizada.
\end{enumerate}\\
\textbf{Poscondición}& Ambos jugadores con la lista de jugadores actualizada.\\
\end{tabularx}
}


\subsection{Enviar Petición de Reto}

{\footnotesize
\begin{tabularx}{0.95\textwidth}{p{0.2\textwidth}|X}

\textbf{Caso de uso} & \textbf{Enviar petición de Reto}\\
& \\
\textbf{Precondición} & Usuarios logeados en el sistea y petición de reto realizada. \\

\textbf{Escenario general} & \begin{enumerate}

\item Al servidor le llega una petición de reto por parte de un usuario.
\item El servidor envía al otro jugador la petición de reto.

\end{enumerate} \\

\textbf{Poscondición} &  \\ \\

\end{tabularx}
}

\subsection{Registrar un movimiento}

{\footnotesize
\begin{tabularx}{0.95\textwidth}{p{0.2\textwidth}|X}

\textbf{Caso de uso} & \textbf{Registrar un movimiento} \\
& \\
\textbf{Precondición} & Iniciar sesión, Unirse a una partida y Conectarse a una
partida \\

\textbf{Escenario general} & \begin{enumerate}
\item Al servidor le llega el movimiento realizado por un usuario.
\item El servidor almacena el movimiento en la base de datos.
\item El servidor notifica al otro usuario que ya puede realizar su movimiento.
\end{enumerate} \\

\textbf{Poscondición} & Se realiza el caso de uso: ''Enviar actualización de la
partida''.\\

\end{tabularx}
}

\subsection{Responder a solicitud de reto}
{\footnotesize
\begin{tabularx}{0.95\textwidth}{p{0.2\textwidth}|X}

\textbf{Caso de uso} & \textbf{Responder a solicitud de reto} \\
& \\
\textbf{Precondición} & Iniciar sesión\\

\textbf{Escenario general} & \begin{enumerate}
\item El servidor envía al usuario la solicitud de reto del otro usuario.
\item El cliente recibe la solicitud y la envía al controlador.
\item El controlador la envía a la ventana de jugadores para que muestre que el jugador ha sido retado
\item El usuario elige la acción en la ventana (si acepta o si no).
\item La ventana de juego envía la información al controlador.
\item La fachada envía la respuesta al servidor por medio del proxy.
\end{enumerate} \\

\textbf{Poscondición} & Respuesta a reto enviada.

\end{tabularx}
}

\subsection{Realizar movimiento}

{\footnotesize
\begin{tabularx}{0.95\textwidth}{p{0.2\textwidth}|X}

\textbf{Caso de uso} & \textbf{Realizar Movimiento} \\
& \\
\textbf{Precondición} & El usuario debe haberse unido a una partida y estar
conectado a ella. \\

\textbf{Escenario general} & \begin{enumerate}
\item El usuario pulsa sobre la casilla del tablero a colocar.
\item La interfaz avisa al controlador del movimiento.
\item El controlador envía el movimiento a la fachada.
\item La fachada envía el movimiento a Tablero9x9.
\item Tablero9x9 comprueba la legalidad del movimiento.
\item Si el movimiento es válido se lo envía al proxy y avisa al controlador para realizar el camino inverso hacia la interfaz y colocar la ficha.
\item Si el movimiento es inválido se lanza una excepción que se recoge en el controlador.
\item El controlador envía la excepción a la interfaz de usuario para comunicarle el error que ha cometido.
\end{enumerate} \\
\textbf{Postcondición} & Movimiento realizado. \\
\end{tabularx}
}



\section{Otros requisitos}
\subsection{Requisitos de la interfaz de usuario}
\subsubsection{Ventana de inicio}
{\footnotesize

 \textbf{Descripción:} Da la bienvenida al usuario.\\ Debe permitir al usuario
ejecutar el
caso de uso ``Iniciar sesión'' e introducir los siguientes datos: \\

\begin{tabularx}{0.9\textwidth}{llX}
\hline
\textbf{Dato} & \textbf{Tipo} & \textbf{Descripción} \\
\hline
NombreUsuario & Texto & Nombre del usuario \\
Contraseña & Texto & Contraseña en el sistema \\
\hline
\end{tabularx}
}

\subsubsection{Ventana de registro}
{\footnotesize



 \textbf{Descripción:} Permite registrarse al usuario.\\
Es necesario que
permita al usuario introducir los siguientes datos: \\

\begin{tabularx}{0.9\textwidth}{llX}
\hline
\textbf{Dato} & \textbf{Tipo} & \textbf{Descripción} \\
\hline
Nombre & Texto & Nombre del usuario \\
eMail & Texto & Dirección de correo electrónico \\
Contraseña & Texto & Contraseña en el sistema \\
Conf. Contraseña & Texto & Confirmación de contraseña \\
\hline
\end{tabularx}

}

\subsubsection{Ventana principal}
{\footnotesize

 \textbf{Descripción:}  Mostrará al usuario las partidas disponibles y las
partidas en
juego.\\
Es necesario que
permita al usuario ejecutar los siguientes casos de uso: \\
\begin{enumerate}
\item Crear una partida.
\item Unirse a una partida.
\item Conectarse a una partida.
\item Ver lista de partidas.

\end{enumerate}
}

\subsubsection{Ventana Crear partida}
{\footnotesize

 \textbf{Descripción:}  Permite al usuario crear una nueva partida.\\
Es necesario que
permita al usuario introducir los siguientes datos: \\

\begin{tabularx}{0.9\textwidth}{llX}
\hline
\textbf{Dato} & \textbf{Tipo} & \textbf{Descripción} \\
\hline
Nombre & Texto & Nombre de la partida \\
Días de juego & Fecha & Días en los que se jugará la partida\\
Hora de Inicio de Juego & Hora & Hora en la que empieza o continua la partida
los días indicados\\
Hora de Fin de Juego & Hora & Hora en la que termina la partida los días
indicados\\
\hline
\end{tabularx}
}

\subsubsection{Ventana de juego}
{\footnotesize



 \textbf{Descripción:}  Muestra al usuario todos los detalles de una partida y
permite
jugar en ella. Deberá mostrar el tablero de juego con la asignación de
territorios.\\

Es necesario que
permita al usuario ejecutar los siguientes casos de uso: \\
\begin{enumerate}
\item Desconectarse de una partida.
\item Realizar un movimiento.
\item Responder acción enemiga.
\item Comprar refuerzos.
\item Enviar petición de alianza.
\item Responder a petición de alianza.
\item Romper alianza.

\end{enumerate}


}
