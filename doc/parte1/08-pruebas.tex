\section{Pruebas de los casos de uso}

\subsection{Pruebas para el servidor}

Las pruebas para el sistema del servidor se han hecho de tal forma que puedan ser ejecutadas de forma independiente. No necesita ningún cliente. El paquete test.communications contiene un cliente construido específicamente para las pruebas y consiste en un modelo simplificado de la clase Cliente del sistema cliente.

\begin{itemize}
\item Paquete \emph{src.server.tests}

\begin{center}
{\footnotesize
\begin{tabularx}{0.95\textwidth}{p{0.2\textwidth}|X}

\textbf{Caso de Uso} & \textbf{Test} \\
& \\
Registrar Usuario & TestDAOJugador\\
Autenticar & TestDAOJugador\\
Cerrar Sesión & TestCerrarSesion\\
Registrar moviminento & TestDAOMovimiento\\
Enviar movimiento a jugador & TestReto\\
Enviar respuesta a reto & TestReto\\
\end{tabularx}
}
\end{center}

Los test no prueban la comunicación entre clientes, solo la funcionalidad del servidor a partir de la fachada del servidor (clase FTERD, ver Diagrama de clases del servidor).

\item Paquete \emph{src.server.tests.communications}.
\begin{center}
{\footnotesize
\begin{tabularx}{0.95\textwidth}{p{0.2\textwidth}|X}
\textbf{Caso de Uso} & \textbf{Test} \\
& \\
Cerrar sesión & TestCerrarSesion\\
Registrar usuario & TestRegistrarYAnyadirJugador\\
Autenticar & TestRegistrarYAnyadirJugador\\
Enviar respuesta a reto & TesteRetoRechazar\\
Enviar respuesta a reto & TestRetoAceptar\\
\end{tabularx}
}
\end{center}
\end{itemize}

\clearpage

\subsection{Pruebas para el cliente}

Las pruebas del cliente son dependientes del servidor. Es decir, el test FTERDTest implica que se esté ejecutando el servidor para la posible evaluación del ``test-case''. Los demás ficheros de test no dependen de la ejecución del servidor, son independientes.


\begin{center}
{\footnotesize
\begin{tabularx}{0.95\textwidth}{p{0.2\textwidth}|X}

\textbf{Caso de Uso} & \textbf{Test} \\
& \\
Registrar & FTERDTest\\
Iniciar sesión & FTERDTest\\
Cerrar sesión &FTERDTest \\
Salir de la partida & FTERDTest\\
Retar jugador &FTERDTest \\
Aceptar reto&FTERDTest\\
Realizar movimiento& Tablero9x9Test\\
\end{tabularx}
}
\end{center}

Existen también los test Tablero3x3Test, JugadorTest y Tablero9x9Test que evalúan los posibles escenarios del juego. Esto es, se evalúan las posibles situaciones de movimientos sobre el tablero y se comprueban los resultados de la ejecución.
\begin{center}
{\footnotesize
\begin{tabularx}{0.95\textwidth}{p{0.2\textwidth}|X}

\textbf{Test} & \textbf{Función} \\
& \\
Tablero3x3Test & Evalúa que se coloquen las fichas en el lugar correcto y el tablero empatado y ganado.\\
Tablero9x9Test & Evalúa la correcta colocación de las fichas según las reglas del juego y que la victoria de los tableros, el empate total y la victoria del juego se realizen correctamente. \\
JugadorTest &Evalúa que el jugador que pone una ficha, realmente ejecuta todo el procedimiento y coloca la ficha, si las reglas lo permiten. \\
\end{tabularx}
}
\end{center}